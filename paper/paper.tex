% Paper template for TAR 2016
% (C) 2014 Jan Šnajder, Goran Glavaš, Domagoj Alagić, Mladen Karan
% TakeLab, FER

\documentclass[10pt, a4paper]{article}

\usepackage{tar2016}

\usepackage[utf8]{inputenc}
\usepackage[pdftex]{graphicx}
\usepackage{booktabs}
\usepackage{amsmath}
\usepackage{amssymb}

% additional packages
\usepackage{enumitem}
\usepackage{makecell}
\usepackage[hyphens]{url}
\usepackage{bm}
\newcommand{\vect}[1]{\bm{#1}}
\newcommand{\matr}[1]{\vect{#1}}
\newcommand{\transpose}{\mathsf T}
	
\title{Token-level author diarization using clustering of stylistic contexts}

\name{Ivan Grubišić, Milan Pavlović, Author3} 

\address{
University of Zagreb, Faculty of Electrical Engineering and Computing\\
Unska 3, 10000 Zagreb, Croatia\\ 
\texttt{\{ivan.grubisic, milan.pavlovic, \}@fer.hr}\\
}
          
         
\abstract{ 
This document provides the instructions on formatting the TAR system description paper in \LaTeX{}. This is where you write the abstract (i.e., summary) of the work you carried out within the project. The abstract is a paragraph of text ranging between 70 and 150 words. This document provides the instructions on formatting the TAR system description paper in \LaTeX{}. This is where you write the abstract (i.e., summary) of the work you carried out within the project. The abstract is a paragraph of text ranging between 70 and 150 words.
}

\begin{document}

\maketitleabstract

\section{Introduction}
In this paper we will focus on the author diarization task proposed on PAN 2016 competition\footnote{\url{http://pan.webis.de/clef16/pan16-web/author-identification.html}}. The aim of this task is to decompose a document into its authorial parts, i.e. to split a text into segments and assign an author to every segment \citep{koppel-2011,aldebei-2015}. This is one of the unsupervised variants of a well known authorship attribution problem since text samples of known authorship are not available \citep{rosso-2016}. As we will describe, in two out of three subtasks of this task only a correct number of authors for a given document is known.

The simplest variant of the authorship attribution problem is about finding the most likely author for a given document from a set of candidate authors whose authentic writing examples are available \citep{stamatatos-2009a,stein-2011,ding-2016}. This problem can be tackled with supervised machine learning techniques as a {{{single-label}}} multiclass text classification problem, where one class represents one author \citep{stamatatos-2009a}. 

The authorship attribution problem is also known as authorship identification and it is a part of authorship analysis \citep{stamatatos-2009a,ding-2016}. Authorship analysis is a field of stylometry and studies information about the authorship of a document, based on features derived from that document \citep{layton-2013}. Moreover, stylometry analyses literary style with statistical methods  \citep{stein-2011}.	

\citet{rosso-2016} divided the PAN 2016 author diarization task into three subtasks. The first subtask is traditionally called intrinsic plagiarism detection (IPD). The goal of this task is to find plagiarized parts of a document in which at least $70\%$ of text is written by main author and the rest by one or more other authors. The term \textit{intrinsic} means that a decision whether plagiarized parts exist has to be made only by analysing a given document, without any comparisons with external sources. In the rest of the paper we refer to this subtask as task \emph{a}.

\begin{table}
	\caption{Basic characteristics of training datasets. * represents that there is a true author and plagiarism segments which don't have to originate from a single author.}
	\label{tab:dataset-characteristics}
	\begin{center}
		\begin{tabular}{cccc}
			\toprule
			Task & \thead{Number of \\ documents} & \thead{Average \\ length (in tokens)} & \thead{(min, max)\\authors} \\
			\midrule
			Task \emph{a} & 71 & 1679 & (2, 2)*\\
			Task \emph{b} & 55 & 3767 & (2, 10)\\
			Task \emph{c} & 54 & 3298 & (2, 10)\\
			\bottomrule
		\end{tabular}
	\end{center}
\end{table}

Other two subtasks are more related to the general task of author diarization. In the second subtask we need to segment a given document and group identified segments by author. In the rest of the paper we refer to the second subtask as a task \emph{b}. The third subtask differs from the second one in the fact that exact number of authors is unknown. In the rest of the paper we refer to the third subtask as a task \emph{c}.

For each of the three subtasks a training dataset is publicly available\footnotemark[1]. \citet{rosso-2016} explain that they are collections of various documents which are part of Webis-TRC-12 dataset \citep{potthast-2013}. Every document in that dataset is constructed from texts of various search results (i.e. authors) for one of the 150 topics in total. By varying different parameters such as the number and proportion of the authors, places in a document where an author switch occurs (between words, sentences or paragraphs), three training and test datasets were generated \citep{rosso-2016}. Test datasets are currently not publicly available and we could not use them for evaluation of our approach. Some basic characteristics of the training datasets are shown in Table \ref{tab:dataset-characteristics}.

% reference+
% authorship analysis+, 
% intrinsic plagiarism detection, author diarization+
% PAN 2016 author diarization task+
% dataset+

\section{Related work}
The basic assumption in authorship analysis is that texts of different authors are mutually separable because each author has a more or less unique writing style \citep{stamatatos-2009a,ding-2016}. More precisely, \citet{koppel-2009} explain that methods used in authorship analysis must be able to distinguish writing styles, but also tolerate shallow differences within the same style because an author's stylistic habits can consciously or unconsciously vary over time. Therefore, most of related work tries to find the better features and methods which writing style will be quantified and measured with.

\citet{zu-2006} manually created a labelled corpus of plagiarized documents and used it for intrinsic plagiarism detection task. They used average sentence length, part of speech tags, average stop word number and the averaged word frequency class as input features for their linear discriminant analysis and support vector machine (SVM) models. They approached that task in a supervised fashion. 

\citet{stamatatos-2009b} created a feature vector of normalized occurrence of character tri-grams in the whole document. That vector represented a document's profile. Using a sliding window of fixed length he created same profiles for every window position and compared them with the profile of the whole document. The result of the comparison was an output from a style change function whose peaks indicate positions in the document where style changes occur. All values above the predefined passage criterion were considered a result of plagiarism. That approach was unsupervised.

\citet{rahman-2015} classified sections of documents from PAN 2011 dataset with the help of SVM. Those sections were again obtained by sliding a window of fixed length over the document. He also proposed a new kind of information theoretical features - entropy, relative entropy, correlation coefficient and $n$-gram frequency class calculated from character tri-gram frequency profiles of each window and the whole document. {{{He also used function word bi-gram and tri-gram frequency profiles with $1$, $2$, $3$ and $4$ skips.}}} The value of the style change function introduced by \citet{stamatatos-2009b} was also incorporated into the feature vectors.

\citet{stein-2011} defined IPD as a one class classification problem where the parts of text written by the main author is associated to the target class and the other parts of text are outliers. To find them, they estimated probability distributions of various stylistic features for the target class and outliers. Then a naive Bayes' algorithm was applied to feature vectors whose values lied outside the predefined uncertainity intervals. Additional outlier post-processing methods were also tested. The most successful was the unmasking technique described by \citet{koppel-2009}. The main sense of unmasking is to iteratively remove the best features that distinguish two classes and observe the speed with which cross-validation accuracy of the again trained classifier drops. If the drop is slow and smooth, the outliers are indeed outliers because after $n$ iterations of removing discriminative features they are still separable from the main auhor's work.

\citet{koppel-2011} used two staged approach in clustering of pre-segmented mixed biblical text written by two authors. First they used normalized cuts algorithm with cosine similarity to obtain initial clusters of segments which were represented only by normalized counts of synonyms from Hebrew synsets. Samples from initial clusers were separated into core and non-core samples via an iterative procedure, and the core ones were labeled. A SVM classifier was used to classify non-core samples, but now represented with bag-of-words feature vectors. The whole approach resulted with very good clusters. They also tried this method on an unsegmented case. The text was first split in a way that minimizes doubly-represented synonyms in segments and the same procedure was repeated. The clustering performance was lower than in pre-segmented case.

\citet{brooke-2013} concluded that a very good initial segmentation of text, at least in poems written by T. S. Elliot, is needed for a good performance of their modified k-means algorithm in clustering of voices. Except common character, lexical and syntactic features, they used features such as average frequency in a large  external corpus \citep{brants-2006}. One of the most promising feature they considered is the centroid of $20$ dimensional distributional vectors obtained by applying latent semantic analysis on a large web corpus \cite{landauer-1997}.

The works by \citet{kuznetsov-2016} and \citet{sittar-2016} were submitted on the PAN 2016 competition for three aforementioned tasks. For the task \emph{a}, \citet{kuznetsov-2016} trained a Gradient Boosting Regression Trees (GBRT) model on PAN 2011 dataset as a style change function used for threshold based outlier detection. Every sentence was vectorized using word frequencies, $n$-gram frequencies, punctuaiton symbols and the universal POS tags count, sentence length and mean length of sentence words. The final input to the model was concatenation of center sentence vector and ones from context of size $\pm2$. In task \emph{b} they used a Hidden Markov Model with Gaussian Emissions for document segmentation over the same sentence scores from task \emph{a}. To estimate the unknown number of authors $n$ in task \emph{c}, they chose $n\in\{2..20\}$ which maximizes their cluster discrepancy measure $Q(n)$.

\citet{sittar-2016} used k-means algorithm to cluster \textit{ClustDist} scores of each sentence, where the number of groups was equal to the known number of authors in tasks \emph{a} and \emph{b}. In task \emph{c}, a number of groups was generated randomly. Although they defined a \emph{ClustDist} score for a single sentence as an average distance between current and every other sentence vector, which is again a similar concept like a style change function, in the provided example they used only a sum of unknown distance measures. $15$ features in total were used for sentence vectorization, including average word and sentence lengths, count and ratios of characters, digits uppercase letters, spaces and tabs. 

The most of the described approaches combine supervised and unsupervised methods and operate on the level of longer text segments or sentences. Since the style change in our tasks can occur even between two tokens in the same sentence, we wanted our model to be able to work on the token level. We were also inspired by \citet{brooke-2013} who said that a more radical approach would not separate the described tasks in segmentation and clustering steps, but rather build authorial segments that would also form good clusters. Instead of clustering tokens directly, we decided to cluster their vectorized stylistic contexts because they obviously contain more valuable stylistic information than tokens alone.

\section{Author diarization and intrinsic plagiarism detetion} \label{sec:author-diraization}
% describe the problem
% define formally: document, segmentation, segment, author assignement (label)
% define evaluation metrics

Let $\Delta$ be the domain of documents. We define a document $D\in \Delta$ as a finite sequence of tokens $(t_i)_{i=1}^n$, where $n$ can differ among documents. Given a document, each of its tokens is unique and defined by its character sequence and position in the document. Therefore, a document can be equivalently represented by its set of tokens $T_D=\{t_i\}_{i=1}^n$.

For each document, there is a corresponding mapping to a sequence of labels $(a_i)_{i=1}^n$ that are representing groupings of tokens by authors. The labels $a_i$ are indices of authors of the document. Each token $t_i\in T_D$ is assigned a document-level label $a_i \in \{1..c\}$ associating it to one of $c$ authors. The exact value of the label is not important. It is only required that all tokens corresponding to the same author have the same label. Therefore, there are $m!$ equivalent such mappings given a document. In the case of intrinsic plagiarism detection, there are only $2$ labels: $0$ representing the main author, and $1$ representing plagiarized text.

Equivalently, the codomain of the mapping can also be defined as a set of segmentations $\Sigma$. A segmentation $S\in \Sigma$ is a minimal set of segments, where each segment $s$ is a set of consecutive tokens $\{t_i\}_{i=i_1}^{i_2}$ where each token is associated with the same author label. For a segmentation to be valid, the segments must cover all terms in the document and not overlap:
\begin{equation}
	\bigcup_{s\in S}s = T_D  \wedge \bigcap_{s\in S}s = \{\}.
\end{equation}
The correct mapping of a documents to the corresponding segmentations will be denoted with $\sigma: \Delta\rightarrow\Sigma$.

Let $\mathcal{D} \subset \Delta\times\Sigma$ be a dataset consisting of a finite set of pairs of documents and corresponding segmentations., i.e. $\mathcal{D} = \{\left(D_i, \sigma(D_i)\right)\}_{i=1}^N$. The goal is to find the model $\hat{\sigma}$ that best approximates the correct mapping $\sigma$, i.e. makes good predictions given unseen documents.

\subsection{Evaluation measures}

For evaluation of intrinsic plagiarism detection, \citet{stein-2010} define multiple measures for different aspects of a system's perfromance. The main measures are binary macro-averaged nad micro-averaged precision ($P$), recall ($R$) and $F_1$-score. For evaluating author diarization, we use \emph{BCubed} precision, recall and $F_1$ measures described by \citet{amigo-2009}, which are specialized for evaluation of clustering results. The same measures were used for evaluation on the PAN 2016 competition \citep{rosso-2016}.

Let $l$ be a function that associates lengths in characters to segments. Specially, $l(\{\}) = 0$. For notational convenience, we also use $l$ to denote the sum of lengths of all segments in a set of segments: $l(S) = \sum_{s\in S} l(s),$ where $S$ is as set of segments. Given a document $D$, let $S_\mathrm{p} \subseteq \sigma(D)$ be a set of all true plagiarism segments of the document and $\hat{S}_\mathrm{p} \subseteq \hat{\sigma}(D)$ the segments predicted as plagiarism by the  model. With ${S_\mathrm{tp} = \bigcup_{(s,\hat{s})\in S_\mathrm{p}\times\hat{S}_\mathrm{p}} l(s\cap\hat{s})}$, the micro-averaged evaluation measures for intrinsic plagiarism detection are defined as follows:
\begin{align}
P_\mathrm{\mu} &= \frac{l(\hat{S}_\mathrm{tp})}{l(\hat{S}_\mathrm{p})}, \\
R_\mathrm{\mu} &= \frac{l(\hat{S}_\mathrm{tp})}{l(S_\mathrm{p})}, \\
F_\mathrm{\mu} &= \frac{2}{P_\mathrm{\mu}^{-1}+R_\mathrm{\mu}^{-1}}.
\end{align}
The macro-average evaluation measures treat all plagiarism segments as equally important and are not affected by their lengths:
\begin{align}
P_\mathrm{M} &= \frac{1}{|\hat{S}_\mathrm{p}|}
	\sum_{\hat{s}\in\hat{S}_\mathrm{p}}
		\frac{{\sum_{s\in S_\mathrm{p}} l(s\cap\hat{s})}}{l(\hat{s})}, \\
R_\mathrm{M} &= \frac{1}{|S_\mathrm{p}|}
	\sum_{\hat{s}\in S_\mathrm{p}}
		\frac{{\sum_{s\in \hat{S}_\mathrm{p}} l(s\cap\hat{s})}}{l(s)}, \\
F_\mathrm{M} &= \frac{2}{P_\mathrm{M}^{-1}+R_\mathrm{M}^{-1}}.
\end{align}

In author diarization document segments have to be clustered into $c$ clusters, where $c$ is the number of authors that may or may not be known to the system. We divide the segments from the true segmentation $S$ and the predicted segmentation $\hat{S}$ each into sets of segments $S_i$, $i=1..c$ and $\hat{S}_j, j=1..\hat{c}$, where $c$ is the true number of authors, and $\hat{c}$ the predicted number of authors. We use the following \emph{BCubed} measures for evaluation:
\begin{align}
P_\mathrm{B^3} &= \sum_{i=1}^c \frac{1}{l(S_i)}\sum_{j=1}^{\hat{c}}
	\sum_{s\in S_i}	\sum_{\hat{s}\in\hat{S}_j} l(s\cap \hat{s})^2 \\
R_\mathrm{B^3} &= \sum_{j=1}^{\hat{c}} \frac{1}{l(\hat{S}_j)}\sum_{i=1}^{c}
	\sum_{s\in S_i}	\sum_{\hat{s}\in\hat{S}_j} l(s\cap \hat{s})^2 \\
F_\mathrm{B^3} &= \frac{2}{P_\mathrm{B^3}^{-1}+R_\mathrm{B^3}^{-1}}.
\end{align}


\section{The proposed approach}
% clustering, sliding window
% pipeline: preprocessing (tokenization) -> basic features -> transformed features (including scaling) -> clustering
% define the pipeline formally P = (f_p: d\times->t, f_b: t->\phi, f_t: \phi->\phi', c: \phi'->labels, f_s: labels\times d->s)
% describe GroupRepel loss, describe AutoKMeans, 
% clustering metric

Our approach can generally be described as a pipeline $P = (f_\mathrm{b}:\Delta\rightarrow \Phi_{n_b}, f_\mathrm{t}:\Phi_{n_b}\rightarrow \Phi_{n_t}, f_\mathrm{c}:\Phi_{n_t}\rightarrow C)$. Here $\Delta$ is the tokenized document domain as defined in section \ref{sec:author-diraization}. $\Phi_{n_b}$ and $\Phi_{n_t}$ are sets of variable-length sequences of feature vectors with dimension $n_b$ and $n_t$ respectively. $n_b$ and $n_t$ can either be fixed or depend on the document being processed. $C$ is the set of all sequences of length $n$ (the number of tokens in the document) with elements being indices of authors/clusters.

The basic feature extractor denoted with $f_\mathrm{b}$ is used to extract stylistic features from the contexts of all tokens. If $D$ is a document with $n$ tokens, the basic feature extractor outputs a sequence of $n$ feature vectors, each vector representing the context of one token. $n_\mathrm{b}$ denotes the dimension of those vectors. The next step in the pipeline is the feature transformation $f_\mathrm{t}$ that maps the basic features to a space that they can be better clustered in. The dimension of the new feature space $n_\mathrm{t}$ generally doesn't equal $n_\mathrm{b}$. The final step in the pipeline is clustering denoted with $f_\mathrm{c}$. The clustering algorithm implicitly clusters tokens because it actually clusters their stylistic contexts, each cluster representing an author. Depending on the task, the clustering algorithm can either be given a known number of authors, or try to predict it.

The following steps are done in predicting the segmentation: (1) raw text is tokenized giving a sequence of tokens $D\in\Delta$, (2) for all tokens, features are extracted from their contexts, giving a sequence of feature vectors $\phi_\mathrm{t} = (\vect{t}_i)_{i=1}^n = (f_\mathrm{t}\circ f_\mathrm{b})(D)$, (3) the tokens are clustered based on $\phi_\mathrm{t}$, giving a sequence of author labels $(a_i)_{i=1}^{n}$, where $n=|T_D|$, and (4) a segmentation $\hat{S}$ is generated based on the document $D$ and the sequence of predicted author labels.

% details for each component
\paragraph{Tokenization.} As a preprocessing step, we tokenized each document $D$ using NLTK\footnote{\url{http://www.nltk.org}} to obtain a set of tokens $T_D$. We also performed part-of-speech tagging with the same tool, to speed up basic feature extraction which we describe below. We didn't use other preprocessing techniques such as lemmatization, stemming and stop word removal because they would take away a lot of stylometric data from text \cite{stamatatos-2009a}. The final output from the tokenization step was a finite sequence $\{(t_i, o_i, l(t_i), \mathit{POS}_i)\}_{i=1}^n$ where $o_i$ is the offset of token $t_i$, $l(t_i)$ its length in characters and $\mathit{POS}_i$ its POS tag.

\paragraph{Basic feature extraction.} 
We defined the stylistic context of a token $t_i$ as a set of tokens $\{t_k\}_{k=i-c}^{i-1} \cup \{t_k\}_{k=i+1}^{i+c}$ where $c$ is a context size. From each context we extracted the most used stylometric features which we found in previous work.
\begin{itemize}
	\item \textit{Character tri-grams}. Frequencies of \textit{n}-grams on character level have been very useful in quantifying the writing style \citep{stamatatos-2009b}. They are able to capture lexical and contextual information, use of punctuation and errors which can be an author's  "fingerprint". This feature is also tolerant to noise. Based on work by \citet{stamatatos-2009a} and \citet{rahman-2015}, we choose $n=3$. Maximal dimension of this feature vector was set to 200.
	\item \textit{Stop words}. According to \citet{stamatatos-2009a}, sometimes also called \textit{function words}, these are the most common used topic-independent words in text, such as articles, prepositions, pronouns and others. They are used unconciously and found to be one of the most discriminative features in authorship attribution since they represent pure stylistic choices of authors' {burrows-1987,argamon-2005}. We used frequencies of 156 english stop words available in \texttt{nltk}.
	\item \textit{Special characters}. We used counts of all character sequences which satisfied a regular expression defined as [REGEX]. Although character n-grams can catch the use of those character sequences, we wanted to have a distinct feature for that purpose. \citet{koppel-2009} mentioned that authors can have different punctuation habbits.
	\item \textit{POS tag counts}. This is syntactic feature which \citet{koppel-2009} and \citet{stamatatos-2009a} also identified as a discriminative one in authorship analysis and it was used by \citet{kuznetsov-2016}. We used all 12 tags from the universal tagset.
	\item \textit{Average token length}. Used by \citet{kuznetsov-2016}, \citet{sittar-2016},  \citet{brooke-2012} and \citet{stein-2011}. \citet{koppel-2009} characterized this feature as a complexity measure.
	\item \textit{Bag of Words}. Bag of words text representation more captures content, rather than style \citep{stamatatos-2009a}. We included this feature because it boosted performance of our initial testing, even without words been previously stemmed or lemmatized (stop words were excluded). Vocabulary had a maximum of 100 words.
	\item \textit{Type-token ratio}. We wanted to use that feature to measure the vocabulary richness in a token's context, but after we finished with performance evaluations we realized that there was a bug in implementation and that feature did not contribute to overal results. The feature should be calculated as the ratio of vocabulary size and total number of tokens of the text \citep{stamatatos-2009a}.
\end{itemize}

Differences between the 2 model variants....

basic features

\paragraph{Feature transformation.} We develop two variants of our model which are mainly differentiated by the application of a trainable feature transformation requiring fixed basic feature space as opposed to variable feature space with features that can be more document-specific.

Let $(\vect{b}_i)_{i=1}^n = f_\mathrm{b}(D)$ be a sequence of basic feature vectors and $(\vect{b}_i')_{i=1}^n$ the corresponding sequence of potentially preprocessed basic feature vectors with elements from $\mathbb{R}^{n_{b}'}$. Let $(a_i)_{i=1}^n$ be the sequence of true author labels with elements from $\{1..c\}$. We want to maximize the \emph{clusterability} of the feature vectors obtained by the feature transformation $T:\mathbb{R}^{n_b'}\rightarrow\mathbb{R}^{n_t}$ with trainable parameters $\vect{\theta}_T$, i.e. we want to maximize segregation and compactness of groups of transformed feature vectors grouped by their target author label. Let $(\vect{t}_i)_{i=1}^n = (T(\boldsymbol{b}_i'))_{i=1}^n = f_\mathrm{t}((\vect{b}_i)_{i=1}^n)$ be the sequence of transformed feature vectors with elements from $\mathbb{R}^{n_t}$. We define the following loss in order to optimize the transformation $T$:
\begin{equation}
	L = \alpha L_\mathrm{c} + (1-\alpha)L_\mathrm{s}
\end{equation}
with $\alpha$ chosen to be $0.5$. Here $L_\mathrm{c}$ is the \emph{compactness loss} and $L_\mathrm{s}$ the \emph{segregation loss}. $L_\mathrm{c}$ is proportional to the average variance of groups, and $L_\mathrm{s}$ penalizes centroids being too close to each other. Let $N_a$ represent the number of tokens associated with author $a$, i.e. $N_a = \sum_{i=1}^n[a_i=a]$, where, for $\varphi$ being a logical proposition $[\varphi]$ evaluates to either $1$ if $\varphi$ is true or $0$ otherwise.
%, the following notation\footnote{It is called \emph{the Iverson bracket}.} is introduced:
%\begin{equation}
%[\varphi] = \begin{cases}
%	1, & \varphi\equiv\top, \\
%	0, & \varphi\equiv\bot.
%\end{cases}
%\end{equation}
Let $\vect{\mu}_a$ represent the current centroid of the transformed feature vectors associated with the author label $a$:
\begin{equation}
\vect{\mu}_a = \frac{1}{N_a}\sum_{i=1}^n T(\vect{b}_i')[a_i=a]
\end{equation}	
We define the components of the loss: 
\begin{align}
	L_\mathrm{c} &= \sum_{a=0}^{c}\frac{1}{N_a} 
		\sum_{i=1}^{n} \| T(\vect{b}_i')-\vect{\mu}_{a}\|^2 [a_i=a], \\
	L_\mathrm{s} &= \frac{2}{c}\sum_{a=0}^{c}\sum_{b=a+1}^{c} \|\vect{\mu}_a-\vect{\mu}_b\|^{-2}.
\end{align}
The \emph{compactness loss} $L_\mathrm{c}$ is a weighted sum of within-group variances. The \emph{segregation loss}  $L_\mathrm{s}$ is proportional to the sum of magnitudes (not magnitude of the sum) of forces between $c$ equally charged particles each located at one of the centroids. By minimizing the average loss (the error) across all documents in the training dataset with respect to the parameters of the transformation $\vect{\theta}_T$ we hope to benefit the clustering algorithm. Instead of squared $L^2$ distance, some other distance measure may be used. Also, the way of combining the two losses was somewhat arbitrarily chosen as well as the segregation loss. For the transformation $T$ we have tried a nonlinear transformations represented as neural networks with one or more hidden layers. We didn't explore it much because it was slower and didn't give as good results as a linear transformation or an elementwise product with a weight vector:
\begin{align}
	T_1(\vect{x}) &= \matr{W}\vect{x}, \quad \vect{\theta}_{T_1} = \matr{W}, \\
	T_2(\vect{x}) &= \vect{w}\odot\vect{x}, \quad \vect{\theta}_{T_2} = \vect{w}.
\end{align}
Before applying the transformations, basic feature vectors are preprocessed by concatanating each with the vector of its squared elements:
\begin{equation}
	\vect{b_i'} = (\vect{b_i}^\transpose, (\vect{b_i}\odot\vect{b_i})^\transpose)^\transpose, \quad i\in \{1..n\}.
\end{equation}
The transformation is implemented using TensorFlow\footnote{TensorFlow link}. It is trained on-line (one document per optimization step) using RMSProp\footnote{RMSProp} with default parameters and learning rate. 


\paragraph{Clustering.}

clustering algorithms

k-means HAC DBSCAN auto-k-means

\paragraph{Segmentation generation.} something short


\section{Experimental results}

For experiments, we use the three PAN 2016 training datasets. The test datasets used in the competition have not been made publicly available. Therefore, our evaluation results might be only moderately comparable to the results of other systems.

We have created two blind baselines and a baseline that tries to really predict the authorial parts based on some simple features. The first blind baseline predicts the whole document to have been written by a single author or, in the case of intrinsic plagiarism detection, labels the whole document as plagiarism. The second blind baseline learns the average lengths of 

\subsection{Experimantal setup}

\subsection{Intrinsic plagiarism detection results}

\subsection{Author diarization results}

1
baselines
repo link
Mention cofidences.

\subsection{Intrinsic plagiarism detection}

setup
results
don't forget to describe tolerance

\subsection{Author diarization with known numbers of authors}

setup

\subsection{Author diarization with unknown numbers of authors}

setup

\section{Conclusion}

Conclusion is the last enumerated section of the paper. It should not exceed half of a column and is typically split into 2--3 paragraphs. No new information should be presented in the conclusion; this section only summarizes and concludes the paper.

\section{Further work}
% more systematic parameter choice, model development
% mention possible applicability to author clustering and extrinsic plagiarism detections

\section*{Acknowledgements}

If suitable, you can include the \textit{Acknowledgements} section before inserting the literature references  in order to thank those who helped you in any way to deliver the paper, but are not co-authors of the paper.

\begin{table}
	\caption{This is the caption of the table. Table captions should be placed \textit{above} the table.}
	\label{tab:narrow-table}
	\begin{center}
		\begin{tabular}{c|ccc|ccc}
			\toprule
			Model & $R_\mathrm{\mu}$ & $P_\mathrm{\mu}$ & $F_\mathrm{\mu}$ & $R_\mathrm{M}$ & $P_\mathrm{M}$ & $F_\mathrm{M}$\\
			\midrule
			Dummy & 0 & 1 & 2 \\
			One & 0 & 1 & 2 \\
			One & 0 & 1 & 2 \\
			\midrule
			One & 0 & 1 & 2 \\
			\bottomrule
		\end{tabular}
	\end{center}
\end{table}

\bibliographystyle{tar2016}
\bibliography{tar2017}

\end{document}

