% Paper template for TAR 2016
% (C) 2014 Jan Šnajder, Goran Glavaš, Domagoj Alagić, Mladen Karan
% TakeLab, FER

\documentclass[10pt, a4paper]{article}

\usepackage{tar2016}

\usepackage[utf8]{inputenc}
\usepackage[pdftex]{graphicx}
\usepackage{booktabs}
\usepackage{amsmath}
\usepackage{amssymb}

\title{TAR System Description Paper Template}

\name{Author1, Author2, Author3} 

\address{
University of Zagreb, Faculty of Electrical Engineering and Computing\\
Unska 3, 10000 Zagreb, Croatia\\ 
\texttt{autor1@xxx.hr}, \texttt{\{autor2,autor3\}@zz.com}\\
}
          
         
\abstract{ 
This document provides the instructions on formatting the TAR system description paper in \LaTeX{}. This is where you write the abstract (i.e., summary) of the work you carried out within the project. The abstract is a paragraph of text ranging between 70 and 150 words.
}

\begin{document}

\maketitleabstract

\section{Introduction}

Author diarization is the problem of segmenting text within a document into classes each corresponding to an author, i.e. assignment of each part of a text to an author. The analyzed document can be a result of collaborative work or plagiarism by a single author. The problem is in more detail described here. The problem can be divided into 3 subproblems:
% reference
% authorship analysis, 
% intrinsic plagiarism detection, author diarization
% PAN 2016 author diarization task
% dataset

\section{Related work}

% Kuznetsov, Sittar, Brooke

\section{The proposed approaches}
clustering, sliding window
pipeline: preprocessing (tokenization) -> basic features -> transformed features (including scaling) -> clustering

clustering metric

\begin{itemize}
\item a
\item b
\end{itemize}

features

differences: fixed features + transformation vs document-dependent features

\section{Experimantal results}

1
baselines

\begin{table}
	\caption{This is the caption of the table. Table captions should be placed \textit{above} the table.}
	\label{tab:narrow-table}
	\begin{center}
		\begin{tabular}{cccc}
			\toprule
			Model & $R$ & $P$ & $F_1$\\
			\midrule
			Dummy & 0 & 1 & 2 \\
			One & 0 & 1 & 2 \\
			One & 0 & 1 & 2 \\
			\midrule
			One & 0 & 1 & 2 \\
			\bottomrule
		\end{tabular}
	\end{center}
\end{table}

Mention cofidences.

\subsection{Intrinsic plagiarism detection}

setup
results

\subsection{Author diarization with known numbers of authors}

setup

\subsection{Author diarization with unknown numbers of authors}

setup

\section{Conclusion}

Conclusion is the last enumerated section of the paper. It should not exceed half of a column and is typically split into 2--3 paragraphs. No new information should be presented in the conclusion; this section only summarizes and concludes the paper.

\section*{Acknowledgements}

If suitable, you can include the \textit{Acknowledgements} section before inserting the literature references  in order to thank those who helped you in any way to deliver the paper, but are not co-authors of the paper.

\bibliographystyle{tar2016}
\bibliography{tar2016} 

\end{document}

